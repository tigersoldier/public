
%%%%%%%%%%%%%%%%%%%%%%%%%%%%%%%%%%%%%%%%%%%%%%%%%%%%%%%%%%%%%%%%%%%%%%%%%%%%%
%   Copyright 2004  by      (FBI@BHQT)
%%%%%%%%%%%%%%%%%%%%%%%%%%%%%%%%%%%%%%%%%%%%%%%%%%%%%%%%%%%%%%%%%%%%%%%%%%%%%
\phantomsection\addcontentsline{toc}{chapter}{\hspace*{0.2cm}Abstract}

\chapter*{ \markboth{Abstract}{Abstract}}

\vspace*{-1.5cm}

\begin{tabular}{ll}
  Title: & \etitle \\
  Major: & \emajor \\
  Name: & \eauthor \\
  Supervisor: & \esupervisor \\
\end{tabular}

\vspace{1.5cm}
\begin{center}
{\hei\xiaoerhao \textbf{Abstract }}
\end{center}


%--------------����------------------%

Network traffic classification is the foundation of many other
network researches. It is meaningful to understand clearly about the
entire traffic statistics of Internet, especially for the research
field of Internet Traffic Modeling, Network Management, Network
Security and Traffic Engineering. Recently, P2P application is
becoming popular and the network traffic generated by P2P
applications reserves a big part of total bandwidth, the
effectiveness of previous port-based and payload-based
classification techniques is diminished. There is a strong need to
propose some advanced and effective classification models to provide
supporting for Internet business, such as QoS Guarantees and Network
Anomaly Detection.



\vspace{1cm} \noindent\textbf{Key Words}: Network traffic
classification; 
